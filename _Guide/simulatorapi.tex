\section{Simulator API Documentation}
The \texttt{ComputerSimulation} class defined in \texttt{\_Simulator/\_src/compSim.h} is the access point to the API. Therefore, all functions described below are member functions of the \texttt{ComputerSimulation} class.
\\ \\
\docBox {\texttt{typedef uint16\_t word}
\\ \small{Definition of word size.}
\\ \\
\texttt{ComputerSimulation(std::string name, std::string loggerFilename)}\\
\small{ Constructs an instance of the class with the given name (not important) and starts logging to the loggerFilename. \textbf{Note} that the simulator will overwrite existing files with the same filename as loggerFilename.}
\\ \\
\texttt{ComputerSimulation(std::string name)}
\\ \small{Same as above but will log to the standard \texttt{logg.txt} filename. }
\\ \\
\texttt{void load(std::string filename)}
\\ \small{Loads and parses sasm-file named \texttt{filename} to the simulator.}
\\ \\
\texttt{void reset()}
\\ \small{Sets the program counter and instruction count to zero, clears breakpoints, wipes instruction- and data memory and resets statistics.}
\\ \\
\texttt{void run()} 
\\ \small{Starts simulation from the program counter value to the last instruction (or next HLT instruction).}
\\ \\
\texttt{bool step()} 
\\ \small{Executes one step (instruction). Returns \texttt{true} if instruction is not HLT.}
\\ \\
\texttt{bool next()} 
\\ \small{Starts simulation. Stops at next breakpoint or HLT instruction. Returns \texttt{true} if last instruction is not HLT.} 
\\ \\
\texttt{void writeToLog(std::string message)}
\\ \small{Write \texttt{message} to log.}
\\ \\
\texttt{void resetStatistics()}
\\ \small{Reset simulator statistics.}
\\ \\
\texttt{std::vector<std::string> memoryDump(word fromAddr, word toAddr, memType memoryType)} 
\\ \small{If \texttt{memoryType == DATA} the returned vector is populated with hexadecimal values from data memory (one word per vector item). If \texttt{memoryType == INSTR} the returned vector is populated with disassembled instructions. The fromAddr and toAddr variables denotes the start and end address of the memory area that is dumped.}  
\\ \\
\texttt{word getPC() const}
\\ \small{Returns the program counter value.}
\\ \\
\texttt{long long getInstructionsSimulated() const}
\\ \small{Returns the number of instructions that has been simulated.}
\\ \\
\texttt{bool isRunning() const}
\\ \small{Returns \texttt{true} if the simulator is in RUN-mode.}
\\ \\
\texttt{bool singleStep() const}
\\ \small{Returns \texttt{true} if the simulator is in SINGLESTEP-mode.}
\\ \\
\texttt{bool validProgram() const}
\\ \small{Returns \texttt{true} if a valid program is loaded to the simulator.}
\\ \\
\texttt{word getInstr(word PC)}
\\ \small{Returns the instruction at the address pointed at by PC (program counter).}
\\ \\
\texttt{Mode getMode() const}
\\ \small{Returns the mode that the simulator is currently in.}
\\ \\
\texttt{std::string getName() const}
\\ \small{Returns simulator name as \texttt{std::string}.}
\\ \\
\texttt{void setBreakpoints(const std::vector<int> &breakpoints)}
\\ \small{Sets breakpoints at the line numbers(in the program file) given by the \texttt{breakpoints} vector.}
\\ \\
\texttt{void setPC(word PC)}
\\ \small{Sets program counter to \texttt{PC}.}
}