\section{Build} \label{bygge}
The project has two user interfaces, one by console and one by graphics. Each of the two has its own directory with its own build configuration. Even though the code for the simulator is in a different directory, both user interfaces depend on it. The two user interfaces, on the other hand, is independent of each other. Therefore, the two UIs should also be built independent of each other. More information on the project structure is available on \href{https://github.com/LasseNatvig/SimComp}{Github}.
\\
\\
Follow the instructions in section (\ref{konsoll}) and (\ref{GUI}) to build the console- and GUI application respectively.

\subsection{Console} \label{konsoll}
The console application only depends on the C++11 standard library. Therefore it can be built as any other C++ project. However, the project provides some convenient files for doing so. If you're using Mac OS or Linux, we will show you how to build the project using \href{https://cmake.org}{\textit{CMake}} and \textit{Make}. On Windows, you can build the project using the provided \texttt{\textunderscore Console/\textunderscore src/SimComp\textunderscore Console.vcxproj} \textit{MSVS} project file.
\subsubsection{Mac OS or Linux}
\begin{enumerate}
    \item Open a \textit{Terminal} window
    \item Change to the project directory:
    \shellcmd{cd <PROJECTPATH>}
    \item Change to \texttt{\textunderscore Console/\textunderscore src/} folder:
    \shellcmd{cd \textunderscore Console/\textunderscore src/}
    \item Generate a \textit{makefile} with \textit{CMake}:
    \shellcmd{cmake .}
    \item Build using \textit{Make}:
    \shellcmd{make} \\
    This will build the project into an executable named "SimComp".
\end{enumerate}

\subsection{GUI} \label{GUI}
There are several ways to build this project. If you're using Windows, we will show you how to build it using \textit{Visual Studio}. On Mac OS or Linux we will show you how to build it using \textit{Qt Creator}. Either way you'll need \href{https://www.qt.io/download}{\textit{Qt}} versio 5.10 or newer. If you're building using \textit{Visual Studio} you'll also need \href{https://marketplace.visualstudio.com/items?itemName=TheQtCompany.QtVisualStudioTools-19123}{\textit{Qt Visual Studio Tools}}.

\subsubsection{Mac OS or Linux}
\begin{enumerate}
    \item Open \textit{Qt Creator}
    \item Navigate to \textit{File} \(\rightarrow\) \textit{Open File or Project} in the menu bar
    \item Navigate to \textit{Build} \(\rightarrow\) \textit{Build All}
    \item To start the application click the green start button in the bottom left corner
\end{enumerate}
\subsubsection{Windows}
\begin{enumerate}
    \item Open \textit{Visual Studio 2017}.
    \item Navigate to \textit{File} \(\rightarrow\) \textit{Open} \(\rightarrow\) \textit{Project/Solution}
    \item Choose \path{SimComp/_GUI/_src/SimComp_GUI.vcxproj}. \textbf{Note} that there may appear a dialog window with information about \textit{Git}. It says that the changes in the console-project not will be tracked by \textit{Git}. Just press ok.
    \item Configuration of \textit{Qt Visual Studio Tools} (Skip if it is already configured):
        \begin{enumerate}
            \item Navigate to \textit{Qt VS Tools} \(\rightarrow\) \textit{Qt Options} in the menu bar. 
            \item Click \textit{Add} in the popup window. 
            \item In the window \textit{Add new Qt version} you'll have to choose the \textit{Qt}-version that you'd like to use. Paste the path or click [...] to choose the folder \path{msvc2017_64} under \path{Qt/<VERSION>/}.
        \end{enumerate}
    \item You can now build the project like any other \textit{Visual Studio}-project.
\end{enumerate}